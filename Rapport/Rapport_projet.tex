\documentclass[12pt]{article}
\usepackage[utf8]{inputenc}
\usepackage[a4paper, margin=2.5cm]{geometry}
\usepackage{graphicx} 
\usepackage{natbib}
\usepackage[french]{babel}

\usepackage[default,scale=0.95]{opensans}
\usepackage[T1]{fontenc}
\usepackage{amssymb} %math
\usepackage{amsmath}
\usepackage{amsthm}
\usepackage{systeme}
\usepackage{bbm}
\usepackage{media9}
\usepackage{makecell}
\usepackage{url}
% \usepackage{lineno}
% \linenumbers

\usepackage{hyperref}
\hypersetup{
    % colorlinks=true,
    % linkcolor=blue,
    % filecolor=magenta,      
    % urlcolor=cyan,
    pdftitle={Projet LRC},
    %pdfpagemode=FullScreen,
    }
\urlstyle{same} %\href{url}{Text}

\renewcommand{\baselinestretch}{1.5}

\begin{document}

\begin{titlepage}
    \begin{center}
        \vspace*{1cm}

        \Huge
        \textbf{Rapport de projet}

        \vspace{0.5cm}
        \LARGE
        Écriture en Prolog d'un démonstrateur basé sur l'algorithme des tableaux pour la logique de description  $ \mathcal{ALC}$

        \vspace{1.5cm}

        \textbf{Charles Vin}\\
        \textbf{Barthelemy Dang-Nhu}

        \vfill



        \normalsize

        \textbf{Année :}
        2022/2023
        \hfill
        \includegraphics[width=0.25\textwidth]{./src/logo.png}
    \end{center}
\end{titlepage}

\tableofcontents
\newpage

\section{Introduction}

\section{Description générale et fonctionnement}
\subsection{Les fichiers}
Pour une meilleure compréhension du code, nous avons séparer celui-ci en plusieurs fichier. Chacun à sa spécialisation : \begin{itemize}
    \item \verb|T-A_box.pl| : C'est ici que l'utilisateur entre la TBox et la ABox initial.
    \item \verb|run.pl| : Contient le prédicat \verb|programme/0|  qui appel les grandes étapes nécessaire à la résolution.
    \item \verb|part1.pl| : Contient les fonctions liées à la première partie décrite par le sujet.
    \item \verb|part2.pl| : Contient les fonctions liées à la deuxième partie décrite par le sujet.
    \item \verb|part3.pl| : Contient les fonctions liées à la troisième partie décrite par le sujet.
    \item \verb|helper.pl| : Contient quelques prédicats utiles
\end{itemize}

\subsection{Utilisation du programme}
\subsubsection{Initialisation de la TBox et de la ABox}

\subsubsection{Lancement et saisie de la proposition à prouver}
Pour lancer le programme entrez cette commande à la racine du projet : \verb|swipl -f run.pl|. Puis dans l'interpréteur prolog utilisez le prédicat \verb|programme| pour lancer le programme.


\subsection{Exemple d'utilisation}

\section{Partie 1 : nom}

\section{Partie 2 : nom}

\section{Partie 3 : nom}

\end{document}